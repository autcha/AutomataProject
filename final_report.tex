%%
%% This is file `sample-sigconf.tex',
%% generated with the docstrip utility.
%%
%% The original source files were:
%%
%% samples.dtx  (with options: `sigconf')
%% 
%% IMPORTANT NOTICE:
%% 
%% For the copyright see the source file.
%% 
%% Any modified versions of this file must be renamed
%% with new filenames distinct from sample-sigconf.tex.
%% 
%% For distribution of the original source see the terms
%% for copying and modification in the file samples.dtx.
%% 
%% This generated file may be distributed as long as the
%% original source files, as listed above, are part of the
%% same distribution. (The sources need not necessarily be
%% in the same archive or directory.)
%%
%% The first command in your LaTeX source must be the \documentclass command.
\documentclass[sigconf]{acmart}

%%
%% \BibTeX command to typeset BibTeX logo in the docs
\AtBeginDocument{%
  \providecommand\BibTeX{{%
    \normalfont B\kern-0.5em{\scshape i\kern-0.25em b}\kern-0.8em\TeX}}}

%% Rights management information.  This information is sent to you
%% when you complete the rights form.  These commands have SAMPLE
%% values in them; it is your responsibility as an author to replace
%% the commands and values with those provided to you when you
%% complete the rights form.
\setcopyright{acmcopyright}
\copyrightyear{2019}
\acmYear{2019}


%%
%% The majority of ACM publications use numbered citations and
%% references.  The command \citestyle{authoryear} switches to the
%% "author year" style.
%%
%% If you are preparing content for an event
%% sponsored by ACM SIGGRAPH, you must use the "author year" style of
%% citations and references.
%% Uncommenting
%% the next command will enable that style.
%%\citestyle{acmauthoryear}

%%
%% end of the preamble, start of the body of the document source.
\begin{document}

%%
%% The "title" command has an optional parameter,
%% allowing the author to define a "short title" to be used in page headers.
\title{Automata \& DNA Computing Project Report}

%%
%% The "author" command and its associated commands are used to define
%% the authors and their affiliations.
%% Of note is the shared affiliation of the first two authors, and the
%% "authornote" and "authornotemark" commands
%% used to denote shared contribution to the research.
\author{Logan Lopez}
\email{loganlopez@mail.usf.edu}

\affiliation{%
  \institution{University of South Florida}
  \streetaddress{4202 E. Fowler Avenue}
  \city{Tampa}
  \state{Florida}
  \postcode{33613}
}

\author{Autumn Chavez}
\email{amchavez@mail.usf.edu}
\affiliation{%
	\institution{University of South Florida}
	\streetaddress{4202 E. Fowler Avenue}
	\city{Tampa}
	\state{Florida}
	\postcode{33613}
}

\author{Carlos Leon}
\email{cleon@mail.usf.edu}
\affiliation{%
	\institution{University of South Florida}
	\streetaddress{4202 E. Fowler Avenue}
	\city{Tampa}
	\state{Florida}
	\postcode{33613}
}

%%
%% By default, the full list of authors will be used in the page
%% headers. Often, this list is too long, and will overlap
%% other information printed in the page headers. This command allows
%% the author to define a more concise list
%% of authors' names for this purpose.

%%
%% The abstract is a short summary of the work to be presented in the
%% article.

%%
%% The code below is generated by the tool at http://dl.acm.org/ccs.cfm.
%% Please copy and paste the code instead of the example below.
%%

\maketitle

\section{Introduction}
The goal for this project was to produce a visually appealing model. The model was to show the translation given a selected input string from the alphabet into a strand of DNA. Then show for each transition which enzymes are used animated in sequence. Lastly, the final generated strand or the failed word and state of the automaton. for the following language:
$$ L = \lbrace (b + ba) * : w \in {a, b}* \rbrace $$
We implemented the Ben Shapiro model, a two state finite state automata using DNA computing. This example is modelled off of the language $ L = \lbrace (b + ba)* \rbrace $.
\begin{center}
\includegraphics[width=0.7\linewidth]{statemachine}
\end{center}
An input string is processed into DNA using set transition rules, spacers, and a terminator. Acceptance or rejection of the input string is determined by taking the created DNA strand and using repeating cycles of the Ligase and FokI enzymes. These enzymes either ligate the input words or cleave the obtained strand, which means the next input word is eliminated. If this process leads to the termination sequence, see Figure 2 and 3, then the input is accepted.
\begin{center}
\includegraphics[width=0.7\linewidth]{transitions}
\end{center}

Overall, our implementation shows the process of evaluating a DNA strand given an input value for the two state finite automata and expands on how it is rejected or accepted for the language.
This project was tested using various input strings and provided correct results.
\begin{center}
	\includegraphics[width=0.7\linewidth]{test}
\end{center}
The Ben Shapiro model of DNA computing has some advantages when compared to traditional DNA sequencing. Mainly in that, that the automaton sequences the relevant DNA/RNA without the need of any super specialized computing power. It does, however, have some downsides. Its lack of complexity, means that for a practical experiment more than one automaton is required. For example, Ben Shapiro did an experiment to detect prostate cancer genes in strand of mRNA using an automaton and had to use three automatons. Furthermore, he his automaton had to do something upon rejection - in this case release a drug suppressant. This is an important point, because the final state of the automaton needs to be checked in order to get the diagnosis, but doing so will require the automaton to release a drug into a system that is detectable. Another downside, is that the transitions are all predetermined in their DNA forms, thus will fail if anything can interfere with a specific binding.

There have been advances in the field of DNA computing in the last two decades beyond that of the simple Ben Shapiro Finite State Machine, however, given the lack of Chemistry and Biology knowledge among our peers, we felt this was a better visual representation. The advances, to describe briefly, push DNA computing from 2 states to 3, use more hardware enzymes, and allow for greater interaction between the enzymes. The latest advancement push DNA computing into Push Down Automata territory. While this may be a great leap, it does rely on a specific sort of DNA where the strands at both ends have been bound together. Furthermore, the stack certainly does not have infinite space, nor can it be accessed without cost. 


\end{document}
\endinput
%%
%% End of file `sample-sigconf.tex'.
